
%========================================================================
%========================== BEAMER options ==============================
%========================================================================

\beamertemplateshadingbackground{blue!5}{structure!5}
\beamertemplatetransparentcovereddynamic
\beamertemplateballitem
\beamertemplatenumberedballsectiontoc

\title[Time-Domain FWI using advanced DG methods]{Time-Domain Full Waveform Inversion using advanced Galerkin Discontinuous methods}
%\title{Petit titre}
\author{Pierre Jacquet}
\institute{Université de Pau et des Pays de l'Adour \\
  INRIA Bordeaux Sud-Ouest, Project-Team Magique 3D \\
Laboratory of Mathematics and its Applications of PAU}
\date{February 25, 2021}


\usetheme{Frankfurt}

\definecolor{myNewColorA}{rgb}{0.85,0.07,0.07}
\definecolor{myNewColorD}{rgb}{0.1, 0.1, 0.3}

\setbeamercolor*{palette primary}{bg=myNewColorA, fg = white}
 \setbeamercolor*{palette secondary}{bg=myNewColorA, fg = white}
 \setbeamercolor*{palette quaternary}{bg=myNewColorD, fg = white}
 \beamertemplatenavigationsymbolsempty

%========================================================================
%========================================================================
%========================================================================


%========================================================================
%==================== PAGE DE GARDE =====================================
%========================================================================


\makeatletter
\newcommand\titlegraphicii[1]{\def\inserttitlegraphicii{#1}}
\centering
\titlegraphicii{}
\setbeamertemplate{title page}
{
  \vbox{}
   {\usebeamercolor[fg]{titlegraphic}\inserttitlegraphic\hfill\inserttitlegraphicii\par}
  \begin{centering}
    \begin{beamercolorbox}[sep=8pt,center]{institute}
      \usebeamerfont{institute}\insertinstitute
    \end{beamercolorbox}
    \begin{beamercolorbox}[sep=8pt,center]{title}
      \usebeamerfont{title}\inserttitle\par%
      \ifx\insertsubtitle\@empty%
      \else%
        \vskip0.25em%
        {\usebeamerfont{subtitle}\usebeamercolor[fg]{subtitle}\insertsubtitle\par}%
      \fi%
    \end{beamercolorbox}%
    \vskip1em\par
    \begin{beamercolorbox}[sep=8pt,center]{date}
      \usebeamerfont{date}\insertdate
    \end{beamercolorbox}%\vskip0.5em
    \begin{beamercolorbox}[sep=8pt,center]{author}
      \usebeamerfont{author}\insertauthor
    \end{beamercolorbox}
  \end{centering}
  %\vfill
}
\makeatother
\author{\textbf{Pierre Jacquet}}
\title[Time-Domain FWI using advanced DG methods]{Time-Domain Full Waveform Inversion using advanced Discontinuous Galerkin methods}
\institute{Université de Pau et des Pays de l'Adour \\
  INRIA Bordeaux Sud-Ouest, Project-Team Magique 3D \\
Laboratory of Mathematics and its Applications of Pau}
\date{February 25, 2021}
  \titlegraphic{\includegraphics[scale=0.25]{inria}}

%========================================================================
%========================================================================
%========================================================================



\pgfplotsset{compat=newest}

%%%%%% TIKZ SET STYLE %%%%%%%%
\tikzset{boxOptions/.style={
    rectangle,
    rounded corners,
    draw=black, very thick,
    text width=9em,
    minimum height=2em,
    text centered}
}

\tikzset{arrowStyle/.style={
    ->,
    thick,
    shorten <=2pt,
    shorten >=2pt}
}

\tikzset{arrowStyleinv/.style={
    <-,
    thick,
    shorten <=2pt,
    shorten >=2pt}
}

\usetikzlibrary{backgrounds}

\newcommand\tikzscale{0cm}
\newlength{\tikzwidth}
\newlength{\tikzheight}

%%%%%%%%%%%%%%%%%%%%%%%%%%%%

% colored hyperlinks
\newcommand{\chref}[2]{%
  \href{#1}{{\usebeamercolor[bg]{AAUsimple}#2}}%
}
\newcommand{\scaption}[1]{\caption{\tiny{#1}}}


\definecolor{light-gray}{gray}{0.95}
\definecolor{myred}{RGB}{200,0,0}
\newcommand{\myred}{red!70!black}
\newcommand{\mygreen}{green!60!black}
\newcommand{\myblue}{blue!40!black}
\newcommand{\code}[1]{\colorbox{light-gray}{\texttt{#1}}}


\newcommand\discreteP{\boldsymbol{\textcolor{\myred}{\bar{P}}}}
\newcommand\discreteV{\boldsymbol{\textcolor{\myred}{\bar{V}}}}
\newcommand\discreteU{\boldsymbol{\textcolor{\myred}{\bar{U}}}}
\newcommand\discreteF{\boldsymbol{\bar{F}}}
\newcommand\discreteG{\boldsymbol{\bar{G}}}

\newcommand\discreteQP{\bar{q_p}}
\newcommand\discreteQV{\bar{q_v}}
\newcommand\discreteQ{\bar{q}}
%\newcommand\discreteD{\bar{d}}
%\newcommand\Lag{\boldsymbol{\mathcal{L}}}


\newcommand\qcqU{\textcolor{\myred}{\widehat{\boldsymbol{{u}}}}}
\newcommand\qcqP{\widehat{p}}
\newcommand\qcqV{\widehat{\textbf{v}}}
\newcommand\contLbd{\boldsymbol{\textcolor{\myblue}{\lambda}}}
\newcommand\qcqLbd{\boldsymbol{\textcolor{\myblue}{\widehat{\lambda}}}}
\newcommand\Lbdun{\boldsymbol{\textcolor{\myred}{\lambda_1}}}
\newcommand\Lbdeux{\boldsymbol{\textcolor{\myred}{\lambda_2}}}
\newcommand\qcqLbdun{\widehat{\lambda_1}}
\newcommand\qcqLbdeux{\boldsymbol{\widehat{\lambda_2}}}

\newcommand\discreteLbd{\textcolor{\myred}{\boldsymbol{\bar{\Lambda}}}}
\newcommand\discreteLbdun{\textcolor{\myred}{\boldsymbol{\bar{\Lambda_1}}}}
\newcommand\discreteLbdeux{\textcolor{\myred}{\boldsymbol{\bar{\Lambda_2}}}}
\newcommand\discreteD{\boldsymbol{\bar{D}}}

%\newcommand\CF{\mathcal{J}}
\newcommand\CFF{\mathcal{G}}
\newcommand\DP{Forward_{\model}}

\newcommand\contAl{\boldsymbol{\alpha}}

%% \newcommand\velocity{v_p}
%% \newcommand\density{\rho_0}
%\newcommand\velocity{\boldsymbol{\textcolor{\mygreen}{c}}}
\newcommand\bulkmodulus{\boldsymbol{\textcolor{\mygreen}{\kappa}}}
%\newcommand\density{\boldsymbol{\textcolor{\mygreen}{\rho}}}


\pgfplotsset{
        colormap/paraview/.style={
                colormap={paraview}{
                        rgb255(0cm)=(20,119,255)
			rgb255(1cm)=(168,181,255)
			rgb255(2cm)=(240,240,240)
			rgb255(3cm)=(255,161,145)
                        rgb255(4cm)=(234,60,53)
                },
        },
}


\newcommand\vectll[4]{\left#1 \begin{array}{c}
    #2\\
    #3\\
  \end{array} \right#4}
