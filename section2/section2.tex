\section{Adjoint Studies}

%%%%%%%%%%%%%%%%%%%%%%%%%%%%%%%%%%%%%%%%%%%%%%%% 0 %%%%%%%%%%%%%%%%%%%%%%%%%%%%%%%%%%%%%%%%%%%% %%%%%%%%%%%%%%%%%%%%%%%%%%%%%%%%%%%%%%%%%%%%%%%%%%%%%%

%%%%%%%%%%%%%%%%%%%%%%%%%%%%%%%%%%%%%%%%%%%%%%%% 1 %%%%%%%%%%%%%%%%%%%%%%%%%%%%%%%%%%%%%%%%%%%%% %%%%%%%%%%%%%%%%%%%%%%%%%%%%%%%%%%%%%%%%%%%%%%%%%%%%%%

\renewcommand\tikzscale{1.3}
\begin{frame}{Adjoint Formulation}
\begin{figure}

\definecolor{color1}{RGB}{255,174,41}   %% myOrange
%\definecolor{color2}{RGB}{216,93,99}  %% myGreen
\definecolor{color3}{RGB}{100,149,237} %% myBlue
\definecolor{color2}{RGB}{223,83,74} %% myRed

\definecolor{colorOne}{RGB}{255,174,41}   %% myOrange
%\definecolor{color2}{RGB}{216,93,99}  %% myGreen
\definecolor{colorThree}{RGB}{100,149,237} %% myBlue
\definecolor{colorTwo}{RGB}{223,83,74} %% myRed


\begin{tikzpicture}[scale=\tikzscale] %% [every node/.style={scale=1}]

\node[boxOptions]
at (0,3.5){ {\textbf{\Large\fontfamily{pzc}\selectfont Continuous \\ Direct Problem}}};

\uncover<2->{
\node[boxOptions]
at (6,3.5){ {\textbf{\Large\fontfamily{pzc}\selectfont Continuous \\ Adjoint Problem}}};

\coordinate (a) at (1.4,3.5);
\coordinate (b) at (4.7,3.5);
\draw[->, >=latex, red!50!white, line width=10pt]   (a) to node[pos=0.4,above]{\small{\textbf{\textcolor{black}{Adjoint}}}} (b) ;
}

\uncover<3->{
\node[boxOptions]
at (6,0.7){\textbf{Discretization of the Continuous Adjoint Problem}};

\draw[arrowStyleinv]
(6,2.1) to[out=90,in=90]
node[sloped,anchor=south]
{}
(6,2.6);

\coordinate (b) at (6,1.2);
\coordinate (a) at (6,3.0);
\draw[->, >=latex, red!50!white, line width=10pt]   (a) to node[fill=colorThree!0,pos=0.3]{\small{\textbf{\textcolor{black}{Discretization}}}} (b) ;
}


\uncover<4->{
\node[boxOptions]
at (0,-0.5){\textbf{Discrete \\Direct Problem}};

%% \draw[arrowStyleinv]
%% (0,0.7) to[out=90,in=90]
%% node[sloped,anchor=south]
%% {\footnotesize{Discretization ~~~~~~~~~~~~~}}
%% (0,2.5);

\coordinate (b) at (0,-0.1);
\coordinate (a) at (0,3.0);
\draw[->, >=latex, blue!50!white, line width=10pt]   (a) to node[fill=colorThree!0]{\small{\textbf{\textcolor{black}{Discretization}}}} (b) ;


\node[boxOptions]
at (6,-0.5){\textbf{Adjoint of the Discrete Problem}};


%% \draw[arrowStyle]
%% (2,0) to[out=0,in=180]
%% node[sloped,anchor=south]
%% {(*)}
%% (4,0);

\coordinate (a) at (1.4,-0.5);
\coordinate (b) at (4.7,-0.5);
\draw[->, >=latex, blue!50!white, line width=10pt]   (a) to node[pos=0.4,below]{\small{\textbf{\textcolor{black}{Adjoint}}}} (b) ;
}

\uncover<5->{
\draw[color=red,line width=2] (4.5,1.4)
rectangle (7.5,-1.0);
}
\end{tikzpicture}
\end{figure}
\end{frame}

%%%%%%%%%%%%%%%%%%%%%%%%%%%%%%%%%%%%%%%%%%%%%%%% 1 %%%%%%%%%%%%%%%%%%%%%%%%%%%%%%%%%%%%%%%%%%%%%%%%%%%%%%



%%%%%%%%%%%%%%%%%%%%%%%%%%%%%%%%%%%%%%%%%%%%%%%% 2 %%%%%%%%%%%%%%%%%%%%%%%%%%%%%%%%%%%%%%%%%%%%%%%%%%%%%%

\begin{frame}{AtD : Adjoint then Discretized Strategy}

  \begin{equation}
    \CF(\contP)=\frac{1}{2}||\textcolor{blue}{d_{obs}} - R\contP||^2
    \end{equation}

  \noindent
  \begin{multicols}{2}
    \noindent
      \begin{empheq}[left=\empheqlbrace]{align}
    & \frac{1}{\density \velocity^2}\frac{\partial \contP}{\partial t}+\nabla \cdot \contV=f_p \text{~~ on $\boldsymbol{\Omega}$}\\
    & \density\frac{\partial \contV}{\partial t}+\nabla\contP=0  \text{~~ on $\boldsymbol{\Omega}$}\\
    & \contP=0 \text{~~ on $\textcolor{red}{\boldsymbol{\Gamma_1}}$} \\
    & \frac{\partial \contP}{\partial t}+\velocity \nabla \contP \cdot \normal=0 \text{~~ on $\textcolor{blue}{\boldsymbol{\Gamma_2}}$}\\
    & \contP(0) = 0 \text{, ~~~} \contV(0) = 0
      \end{empheq}
      \vspace{30cm}
    \columnbreak
    \noindent
      \begin{empheq}[left=\empheqlbrace]{align}
    & \frac{1}{\density \velocity^2}\frac{\partial \Lbdun}{\partial t}+\nabla \cdot \Lbdeux=\frac{\partial \CF}{\partial \contP} \text{~~ on $\boldsymbol{\Omega}$}\\
    & \density\frac{\partial \Lbdeux}{\partial t}+\nabla\Lbdun=0  \text{~~ on $\boldsymbol{\Omega}$}\\
    & \Lbdun=0 \text{~~ on $\textcolor{red}{\boldsymbol{\Gamma_1}}$} \\
    & \frac{\partial \Lbdun}{\partial t}-\velocity \nabla \Lbdun \cdot \normal=0 \text{~~ on $\textcolor{blue}{\boldsymbol{\Gamma_2}}$}\\
    & \Lbdun(T) = 0 \text{, ~~~} \Lbdeux(T) = 0
  \end{empheq}

  \end{multicols}
  \vspace{-0.5cm}
  \begin{equation}
    t\in[0,T] \text{~~~~~~~~~~~~~~~~~~~~~~~~~~} t\in[T,0]
    \end{equation}
\end{frame}

%%%%%%%%%%%%%%%%%%%%%%%%%%%%%%%%%%%%%%%%%%%%%%%% 3 %%%%%%%%%%%%%%%%%%%%%%%%%%%%%%%%%%%%%%%%%%%%%%%%%%%%%%

\subsection{Adjoint then Discretized}
\begin{frame}{AtD : Adjoint then Discretized Strategy}

  \begin{equation}
    \CF(\contP)=\frac{1}{2}||\textcolor{blue}{d_{obs}} - R\contP||^2
  \end{equation}

  \noindent
  \begin{multicols}{2}
    \noindent
    \begin{empheq}[left=\empheqlbrace]{align}
  & \frac{\partial \discreteU}{\partial t}^n=A\discreteU^n+ \discreteF^n \\[0.2cm]
  & \text{With : ~~}  \discreteU^n=\vectll{(}{\discreteP^n}{\discreteV^n}{)}
    \end{empheq}
    \vspace{0.3cm}
    \begin{figure}
      \noindent
       \begin{tikzpicture}[scale=0.8]
      \draw[color=black,line width=2.1](0.0,0.0) -- (5,0.0);
      %\draw[color=blue, line width=10] (0,-0.02) node {$\bullet$} ;
      %\draw[color=blue, line width=10] (5,-0.02) node {$\bullet$} ;
     % \draw node[color=blue,fill,circle,minimum size=0.01](1,1) {};
      \node[anchor=south east, color=black]
      at (0,0) {$0$};
      \node[anchor=south west, color=black]
      at (5,0) {$T$};
      
      \pgfmathsetmacro{\x}{0.0}
      \draw[color=black,line width=2.1](\x,0.1) -- (\x,-0.1);

      \pgfmathsetmacro{\x}{5.0}
      \draw[color=black,line width=2.1](\x,0.1) -- (\x,-0.1);

      \pgfmathsetmacro{\x}{0.5}
      \draw[color=black,line width=1.5](\x,0.1) -- (\x,-0.1);
      \draw[arrowStyle,color=blue]
      (\x-0.5,0) to[out=90,in=90,looseness=4.0]
      node[sloped,anchor=south]
      {}
      (\x,0.0);
      \draw[color=black,line width=1.5](\x,0.1) -- (\x,-0.1);


      
      \pgfmathsetmacro{\x}{1.0}
            \draw[arrowStyle,color=blue]
      (\x-0.5,0) to[out=90,in=90,looseness=4.0]
      node[sloped,anchor=south]
      {}
      (\x,0.0);
      \draw[color=black,line width=1.5](\x,0.1) -- (\x,-0.1);
      \pgfmathsetmacro{\x}{1.5}
            \draw[arrowStyle,color=blue]
      (\x-0.5,0) to[out=90,in=180,looseness=1.0]
      node[sloped,anchor=south]
      {}
      (\x,0.55);

      
      \draw[color=black,line width=1.5](\x,0.1) -- (\x,-0.1);
      \pgfmathsetmacro{\x}{2.0}
      \draw[color=black,line width=1.5](\x,0.1) -- (\x,-0.1);
      \pgfmathsetmacro{\x}{2.5}
      \draw[color=black,line width=1.5](\x,0.1) -- (\x,-0.1);
      \pgfmathsetmacro{\x}{3.0}
      \draw[color=black,line width=1.5](\x,0.1) -- (\x,-0.1);
      \pgfmathsetmacro{\x}{3.5}
      \draw[color=black,line width=1.5](\x,0.1) -- (\x,-0.1);
      \pgfmathsetmacro{\x}{4.0}
      \draw[color=black,line width=1.5](\x,0.1) -- (\x,-0.1);
      \pgfmathsetmacro{\x}{4.5}
      \draw[color=black,line width=1.5](\x,0.1) -- (\x,-0.1);
    \end{tikzpicture}

Time-steps going Forward
    \end{figure}
    \columnbreak
    \noindent
    \begin{empheq}[left=\empheqlbrace]{align}
   \boldsymbol{~~~}   & \frac{\partial \discreteLbd}{\partial t}^n=A\discreteLbd^n+R^*(R\discreteU^n-\textcolor{blue}{d_{obs}})\\
  & \text{With : ~~}  \discreteLbd^n=\vectll{(}{\discreteLbdun^n}{\discreteLbdeux^n}{)}
    \end{empheq}
    \vspace{-0.0cm}
    \noindent
    \begin{figure}
      \noindent
       \begin{tikzpicture}[scale=0.8]
      \draw[color=black,line width=2.1](0.0,0.0) -- (5,0.0);
      %\draw[color=blue, line width=10] (0,-0.02) node {$\bullet$} ;
      %\draw[color=blue, line width=10] (5,-0.02) node {$\bullet$} ;
     % \draw node[color=blue,fill,circle,minimum size=0.01](1,1) {};
      \node[anchor=south east, color=black]
      at (0,0) {$0$};
      \node[anchor=south west, color=black]
      at (5,0) {$T$};
      
      \pgfmathsetmacro{\x}{0.0}
      \draw[color=black,line width=2.1](\x,0.1) -- (\x,-0.1);

      \pgfmathsetmacro{\x}{5.0}
      \draw[color=black,line width=2.1](\x,0.1) -- (\x,-0.1);

      \pgfmathsetmacro{\x}{0.5}
      \draw[color=black,line width=1.5](\x,0.1) -- (\x,-0.1);
      \draw[color=black,line width=1.5](\x,0.1) -- (\x,-0.1);


      
      \pgfmathsetmacro{\x}{1.0}
      \draw[color=black,line width=1.5](\x,0.1) -- (\x,-0.1);
      \pgfmathsetmacro{\x}{1.5}
      \draw[color=black,line width=1.5](\x,0.1) -- (\x,-0.1);
      \pgfmathsetmacro{\x}{2.0}
      \draw[color=black,line width=1.5](\x,0.1) -- (\x,-0.1);
      \pgfmathsetmacro{\x}{2.5}
      \draw[color=black,line width=1.5](\x,0.1) -- (\x,-0.1);
      \pgfmathsetmacro{\x}{3.0}
      \draw[color=black,line width=1.5](\x,0.1) -- (\x,-0.1);
      \pgfmathsetmacro{\x}{3.5}


      \draw[arrowStyle,color=blue]
      (\x+0.5,0) to[out=90,in=0,looseness=1.0]
      node[sloped,anchor=south]
      {}
      (\x,0.55);
      

      
      \draw[color=black,line width=1.5](\x,0.1) -- (\x,-0.1);
      \pgfmathsetmacro{\x}{4.0}


      \draw[arrowStyle,color=blue]
      (\x+0.5,0) to[out=90,in=90,looseness=4.0]
      node[sloped,anchor=south]
      {}
      (\x,0.0);
      
      
      \draw[color=black,line width=1.5](\x,0.1) -- (\x,-0.1);
      \pgfmathsetmacro{\x}{4.5}

      \draw[arrowStyle,color=blue]
      (\x+0.5,0) to[out=90,in=90,looseness=4.0]
      node[sloped,anchor=south]
      {}
      (\x,0.0);
      
      \draw[color=black,line width=1.5](\x,0.1) -- (\x,-0.1);
    \end{tikzpicture}

      Time-steps going Backward
    \end{figure}
  \end{multicols}
\end{frame}
%%%%%%%%%%%%%%%%%%%%%%%%%%%%%%%%%%%%%%%%%%%%%%%% 3 %%%%%%%%%%%%%%%%%%%%%%%%%%%%%%%%%%%%%%%%%%%%%%%%%%%%%%



%%%%%%%%%%%%%%%%%%%%%%%%%%%%%%%%%%%%%%%%%%%%%%%% 4 %%%%%%%%%%%%%%%%%%%%%%%%%%%%%%%%%%%%%%%%%%%%%%%%%%%%%%
\subsection{Discretize then Adjoint}
\begin{frame}{DtA : Discretize then Adjoint Strategy}{RK4 example}

  All time scheme can be summed-up such as :
  \begin{equation}
    \textcolor{\myblue}{\boldsymbol{L}}\discreteU=\textcolor{\myblue}{\boldsymbol{E}}\discreteF
  \end{equation}

  \small
      RK4 time-scheme leads to :
    \begin{equation}
      \discreteU^{n+1}=B\discreteU^n+\textcolor{\myblue}{\boldsymbol{C_0}}\discreteF^n+\textcolor{\myblue}{\boldsymbol{C_{\frac{1}{2}}}}\discreteF^{n+\frac{1}{2}}+\textcolor{\myblue}{\boldsymbol{C_1}}\discreteF^{n+1}
    \end{equation}

\begin{equation}
  \textcolor{\myblue}{\boldsymbol{L}}\discreteU=\textcolor{\myblue}{\boldsymbol{E}}\discreteF=\discreteG
\end{equation}
\begin{equation}
  \begin{pmatrix}
    I & & & & \\
    -B&I & & & \\
    & -B&I  & & \\
    & & \ddots & \ddots   & \\
    & &  & -B &I \\
    %% \vdots & \ddots & \vdots \\
    %% 0      & \cdots & 1
  \end{pmatrix}
    \begin{pmatrix}
    \discreteU^0 \\
    \discreteU^1 \\
    \discreteU^2 \\
    \vdots \\
    \discreteU^n \\
  \end{pmatrix}=
  \begin{pmatrix}
    \discreteG^0 \\
    \discreteG^1 \\
    \discreteG^2 \\
    \vdots \\
    \discreteG^n \\
  \end{pmatrix}
  \end{equation}
\end{frame}


%%%%%%%%%%%%%%%%%%%%%%%%%%%%%%%%%%%%%%%%%%%%%%%% 4 %%%%%%%%%%%%%%%%%%%%%%%%%%%%%%%%%%%%%%%%%%%%%%%%%%%%%%
\begin{frame}{DtA : Discretize then Adjoint Strategy}

    All time scheme can be summed-up such as :
    \begin{equation}
      \textcolor{\myblue}{\boldsymbol{L}}\discreteU=\textcolor{\myblue}{\boldsymbol{E}}\discreteF \uncover<2->{=\discreteG}
    \end{equation}
    We are looking for a Discrete Adjoint state satisfying :
    \begin{equation}
      \textcolor{\myblue}{\boldsymbol{L^*}}\discreteLbd=-R^*(\textcolor{blue}{d_{obs}}-R\discreteU) \uncover<2->{=\discreteD}
    \end{equation}
    With the adjoint operator $\textcolor{\myblue}{\boldsymbol{L^*}}$ satisfying :
      \begin{equation}
    <\textcolor{\myblue}{\boldsymbol{L}}\discreteU,\discreteLbd>=<\discreteU,\textcolor{\myblue}{\boldsymbol{L^*}}\discreteLbd>
      \end{equation}
      \uncover<2->{
        \begin{equation}
        <\discreteG,\discreteLbd>=<\discreteU,\discreteD> \text{~~~(Adjoint Test)}
      \end{equation}


      \begin{center}
        Adjoint test succeeds $\Longleftrightarrow$  operator $\textcolor{\myblue}{\boldsymbol{L^*}}$  well established
      \end{center}
      }
\end{frame}
%%%%%%%%%%%%%%%%%%%%%%%%%%%%%%%%%%%%%%%%%%%%%%%% 4 %%%%%%%%%%%%%%%%%%%%%%%%%%%%%%%%%%%%%%%%%%%%%%%%%%%%%%

%% %%%%%%%%%%%%%%%%%%%%%%%%%%%%%%%%%%%%%%%%%%%%%%%% 5 %%%%%%%%%%%%%%%%%%%%%%%%%%%%%%%%%%%%%%%%%%%%%%%%%%%%%%
%% \begin{frame}{DtA : Discretize then Adjoint Strategy}{Example with RK4}
%% \small
%%       RK4 time-scheme leads to :
%%     \begin{equation}
%%       \discreteU^{n+1}=B\discreteU^n+C_0\discreteF^n+C_{\frac{1}{2}}\discreteF^{n+\frac{1}{2}}+C_1\discreteF^{n+1}
%%     \end{equation}

%% \begin{equation}
%%   L\discreteU=E\discreteF=\discreteG
%% \end{equation}
%% \begin{equation}
%%   \begin{pmatrix}
%%     I & & & & \\
%%     -B&I & & & \\
%%     & -B&I  & & \\
%%     & & \ddots & \ddots   & \\
%%     & &  & -B &I \\
%%     %% \vdots & \ddots & \vdots \\
%%     %% 0      & \cdots & 1
%%   \end{pmatrix}
%%   \begin{pmatrix}
%%     \discreteU^0 \\
%%     \discreteU^1 \\
%%     \discreteU^2 \\
%%     \vdots \\
%%     \discreteU^n \\
%%   \end{pmatrix}=
%%   \begin{pmatrix}
%%     \discreteG^0 \\
%%     \discreteG^1 \\
%%     \discreteG^2 \\
%%     \vdots \\
%%     \discreteG^n \\
%%   \end{pmatrix}
%% \end{equation}

%% So :

%% \begin{equation}
%%   L^*=\begin{pmatrix}
%%   I &-B^* & & & \\
%%   &I &-B^* & & \\
%%   & &\ddots  &\ddots & \\
%%   & &  & I   &-B^* \\
%%   & &  &  &I \\
%%   %% \vdots & \ddots & \vdots \\
%%   %% 0      & \cdots & 1
%%   \end{pmatrix}
%% \end{equation}
%% \end{frame}

\begin{frame}{Adjoint Strategies Comparison}
  \begin{columns}
    \begin{column}[t]{0.5\textwidth}
      \textbf{\textcolor{red}{Adjoint Then Discretize}}
      \vspace{0.5cm}
%      \dotfill % to show column margins
      \begin{itemize}
      \item[\textcolor{\mygreen}{\textbf{+}}] Physical approach
      \item[\textcolor{\mygreen}{\textbf{+}}] Same discrete operators for Forward and Backward
      \item[\textbf{- -}] Inexact gradient \cite{Sirkes}
      \end{itemize}
      %      \dotfill
      \vspace{0.5cm}
    \end{column}\vrule \hfill
    \begin{column}[t]{0.5\textwidth}
      \textbf{\textcolor{blue}{Discretize then Adjoint}}
      \vspace{0.5cm}
%            \dotfill
      \begin{itemize}
      \item[\textcolor{\mygreen}{\textbf{+}}] Numerical approach
      \item[\textcolor{\mygreen}{\textbf{+}}] Has an Adjoint Test
      \item[\textbf{-}] Tremendous work to develop the adjoint operators
      \item[\textcolor{black}{\textbf{?}}] Non-consistency of the adjoint state \cite{Set1997Feb}
      \end{itemize}
    \end{column}
  \end{columns}

  \vfill
  \tiny
  \begin{thebibliography}{2}
    \bibitem{Sirkes} Sirkes, Ziv and Tziperman, Eli
      \newblock Finite Difference of Adjoint or Adjoint of Finite Difference ?
      \newblock 1997
  \bibitem{Set1997Feb} Sei Alain and Symes William
    \newblock A Note on Consistency and Adjointness for Numerical Schemes
    \newblock 1997
  \end{thebibliography}

\end{frame}
